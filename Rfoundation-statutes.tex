\documentclass[a4paper]{article}

\title{\bf Statutes of The R Foundation for Statistical Computing}
\date{Draft, \today}

\usepackage{url}

\newcommand{\note}[2]{{\it \par\noindent\textbf{#1:} #2}}

\begin{document}

\maketitle

\it

\section*{Foreword}

The structure of this statutes is modeled after a template provided by
the Austrian Ministry of the Interior (which is responsible for
admission of non-profit associations). Following the template is a
good idea because many Austrian associations use similar statutes and we
will probably need no legal advice for passing them. Other sources I
have used are the statutes of
the Apache Foundation (\url{http://www.apache.org}),
GNOME Foundation (\url{http://www.gnome.org}),
IASC (\url{http://www.iasc-isi.org}),
the Austrian Statistical Society (\url{http://www.osg.or.at}),
and the Linux User Group Austria (\url{http://www.luga.or.at});
the latter two mostly to have examples
for Austrian associations with similar objectives.

All text in italics (including this foreword) are comments and will
not appear in the final document. The document we submit to the
Austrian authorities must be in German, Kurt and I will translate it
once we have an agreement on the English version. Everything below is
of course only meant as a starting point for discussions, and a lot of
it will be useful even if we establish the foundation not in Austria.

\medskip\noindent
Fritz

\medskip\noindent
PS as usual: feel free the improve the wording at any time, the
vocabulary needed here is not exactly the same as in scientific
papers (and following a German template usually also does not exactly
help to improve my English).

\rm

\section{Name, Seat and Field of Activity}

\begin{enumerate}
 \item The organization is named ``The R Foundation for Statistical
  Computing'', abbreviated as ``R Foundation'', which will be used
  throughout this document.
 \item It is seated in Vienna, Austria, and is active worldwide.
\end{enumerate}

\note{FL}{We will use the English name also in the German version and
  register the association with it, i.e., the English name will be the
  registered official name.}

\section{Objectives}

\begin{enumerate}
 \item Fundamentals
  \begin{enumerate}
   \item The activities of the ``R Foundation'' and its members are not
    influenced by politics or religion; people may become members
    regardless of nationality, religion, or place of residency.
   \item The ``R Foundation'' is a non-profit organization working in
    the public interest.
  \end{enumerate}
 \item The objectives of the ``R Foundation'' are:
  \begin{enumerate}
   \item Advance the R project for statistical computing to provide a
    free and open source software environment for data analysis and
    graphics. 
   \item Act as an official voice for the R project, providing means of
    communication with the press, commercial and noncommercial
    organizations interested in R.
   \item Hold and administer the copyright of the R software and
    documentation.
  \end{enumerate}

 \item To meet these objectives the organization will especially
  \begin{enumerate}
   \item Support the further development of R and related open source
    software projects.
   \item Coordinate research projects, support communication within
    the R user community, and organize or sponsor courses.
   \item Operate Internet information systems like email, FTP or HTTP
    Servers.
   \item Organize and sponsor R-related conferences and workshops,
    represent R at relevant conferences sponsored by others, and
    promote the use and development of R and R-related software.
   \item Publish manuals, technical standards, periodicals, journal
    articles and other R-related documents in printed and electronic
    form.
  \end{enumerate}

 \item The material means of the organization to meet its objectives will
  come from membership fees, donations, contributions and registration
  fees. 
  
\end{enumerate}

\note{FL}{Most of the above has been adopted from the GNOME charta.
  Basically we have to list everything we may what to do (and spend
  money for), not what we actually will do. For activities that are
  our core purposes, i.e., listed above, we don't have to pay taxes,
  even if they generate income, like organizing a conference.}

\section{Membership}

\begin{enumerate}
 \item The ``R Foundation'' consists of ordinary and supporting
  members.  Ordinary members have a vote in the general assembly and
  participate actively in the work of the organization. Supporting
  members have no vote and promote the organization primarily by
  paying membership fees.
  
 \item Only people can become ordinary members. New ordinary
  members shall be admitted only by a majority vote of the existing
  ordinary members. This vote can be conducted either at a general
  assembly of the ``R Foundation'' or by electronic means.  The
  initial set of ordinary members at establishment of the organization
  will consist of the members of the ``R Development Core Team'' as
  listed in the sources of R release 1.5.0.
  
 \item Any person or legal entity may become a supporting member.  New
  supporting members can be temporarily admitted by the board of the
  organization.  This temporary admission must be approved by the
  general assembly.  Approval for membership can be rejected without
  public justification.

 \item Membership terminates
  \begin{enumerate}
   \item at the death of a person or the termination of
    existence of legal entities.
   \item by voluntary withdrawal from membership through written
    notice to the board of the organization.
   \item by an affirmative vote of a two-thirds majority of the
    ordinary members. 
  \end{enumerate}
  
\end{enumerate}
   

\section{General Assembly}

\note{FL}{To be written. Idea is to have a general assembly at least
  once every 2 (???) years somewhere on the globe and make Internet
  votes the usual form for getting decisions.  Every ordinary member
  has one vote, the GA elects the board and 2 additional auditors,
  fixes membership fees (different for individuals and institutions),
  approves the bookkeeping, may change the statutes, \ldots}

\section{Board}

\begin{enumerate}
 \item The board of the organziation consists of at least four
  persons:
  \begin{enumerate}
   \item Either a president and a vice-president or two presidents of
    equal rights.
   \item A secretary.
   \item A treasurer.
  \end{enumerate}
  Optionally a vice-secretary and a vice-treasurer may be nominated if
  necessary.
\end{enumerate}

\note{FL}{Having two presidents is of course a ``Lex R'', but I think it
  would be great to start initially with R\&R as presidents. Details
  have to be written yet. The presidents represent the foundation, the
  secretary runs the daily business and the treasurer handles the
  money. Larger financial transactions need the OK from two board members.}

\section{Auditors}

\note{FL}{2 non-board-members}

\section{Court of Arbitration}

\note{FL}{to be written}

\section{Termination}

\note{FL}{we have to decide where assets go to in case we terminate the
  foundation, general targets like ``similar open source projects''
  are fine. should be such that we can easily transfer assets to another
  R foundation in case we want to move to another country.}


\end{document}

%%% Local Variables: 
%%% mode: latex
%%% TeX-master: t
%%% End: 
